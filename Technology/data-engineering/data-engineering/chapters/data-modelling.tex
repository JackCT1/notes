\chapter{Data Modelling}

\textbf{Relational database} - A database that stores data in tables of rows and columns.\\

Each table in a RD represents a aspecific object like a 'customers' table for example.
They're called relational because entities (tables) can have relationships with others, connecting via a
common field (column).

\subsubsection*{Advantages of Relational Databases}

\begin{itemize}
    \item Flexible - easy to make changes without changing overall structure
    \item Built-in security
    \item Easy to run complex SQL queries for analysis - joins and aggregates etc
    \item ACID compliant
\end{itemize}

\textbf{ACID} transactions - Atomicity, Consistency, Isolation, Durability\\

\textbf{Atomicity} - all queries in a transaction must succeed for the transaction to succeed. 
If one query fails, the entire transaction fails.\\

\textbf{Consistency} - Referenital integrity. Data kept consistent throughout the tables
through the use of foreign keys.\\

\textbf{Isolation} - Transactions running simultaneously do not interfere with each other.\\

\textbf{Durability} - Guarantee that any changes made by a committed transaction are not lost.

\subsubsection*{1st Normal Form}

\begin{itemize}
    \item Atomic Values: Each cell contains a single value.
    \item There must be a primary key to identify rows.
    \item No duplicate rows or columns.
\end{itemize}

\subsubsection*{2nd Normal Form}

\begin{itemize}
    \item Must be in 1st normal form
    \item Each non-key field must be fully dependent on the primary key
\end{itemize}

\subsubsection*{3rd Normal Form}

\begin{itemize}
    \item Must be in 2nd normal form
    \item Each non-key field must not be dependent on another.
\end{itemize}